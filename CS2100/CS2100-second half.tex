\documentclass[12pt, landscape]{article}
\usepackage[scaled=0.92]{helvet}
\usepackage{multicol}
\usepackage{calc}
\usepackage{ifthen}
\usepackage[landscape]{geometry}
%\usepackage{hyperref}

\usepackage{newtxtext} 

%for strikeout
\usepackage{ulem}

%For editing parbox
\usepackage[table]{xcolor}
%For editing itemise margins, reduce iterm separaion and list separation
\usepackage{enumitem}
% For math
\usepackage{amsmath,amsthm,amsfonts,amssymb}

%For pictures / figures
\usepackage{color,graphicx,overpic}
\graphicspath{ {./images/} }

%\usepackage{newtxtext} 
%\usepackage{amssymb}
%\usepackage[table]{xcolor}
%\usepackage{vwcol}
%\usepackage{tikz}
%\usepackage{wrapfig}
%\usepackage{makecell}


% For Code Blocks
\usepackage{xcolor}
\usepackage{listings}

% C++ Code Blocks:
\definecolor{listinggray}{gray}{0.9}
\definecolor{lbcolor}{rgb}{0.9,0.9,0.9}
\definecolor{Darkgreen}{rgb}{0,0.4,0}
\lstset{
    backgroundcolor=\color{lbcolor},
    tabsize=4,    
%   rulecolor=,
    language=[GNU]C++,
        basicstyle=\tiny,
        upquote=true,
        aboveskip={0.5\baselineskip},
	% Represents top margin
        columns=fixed,
        showstringspaces=false,
        extendedchars=false,
        breaklines=true,
        prebreak = \raisebox{0ex}[0ex][0ex]{\ensuremath{\hookleftarrow}},
        frame=single,
	% Remove Numbers
        numbers=none,
        showtabs=false,
        showspaces=false,
        showstringspaces=false,
        identifierstyle=\ttfamily,
        keywordstyle=\color[rgb]{0,0,1},
        commentstyle=\color[rgb]{0.026,0.112,0.095},
        stringstyle=\color[rgb]{0.627,0.126,0.941},
        % numberstyle=\color[rgb]{0.205, 0.142, 0.73},
%        \lstdefinestyle{C++}{language=C++,style=numbers}’.
}
\lstset{
    backgroundcolor=\color{lbcolor},
    tabsize=4,
  language=C++,
  captionpos=b,
  tabsize=3,
  frame=lines,
  % Remove Numbers
  numbers=none,
  numberstyle=\tiny,
  numbersep=3 pt,
  breaklines=true,
  showstringspaces=false,
  basicstyle=\tiny,
%  identifierstyle=\color{magenta},
  keywordstyle=\color[rgb]{0,0,1},
  commentstyle=\color{Darkgreen},
  stringstyle=\color{red}
}
% \lstset{language=C++}
%% Different languages: SQL, C++, Python


%Helpful:
%[linewidth = 1.0 \linewidth]
%\lstinline{}
% use \code{} for \lstinline with colorbox.
\newcommand{\code}[1]{\colorbox{gray!25!}{\lstinline[basicstyle=\scriptsize]|#1|}}

% Template: Cheatsheet with code enabled

%--------------------------- PACKAGES ABOVE --------------------------------------------------------------

\pdfinfo{
  /Title (CS2100 Summary.pdf)
  /Creator (Ger Teck)
  /Author (Ger Teck)
  /Subject ()
  /Keywords (tex)}

%% Margins for PAPER

% This sets page margins to .5 inch if using letter paper, and to 1cm
% if using A4 paper. (This probably isn't strictly necessary.)
% If using another size paper, use default 1cm margins.
\ifthenelse{\lengthtest { \paperwidth = 11in}}
	{ \geometry{top=.5in,left=.5in,right=.5in,bottom=.5in} }
	{\ifthenelse{ \lengthtest{ \paperwidth = 297mm}}
		{\geometry{top=1cm,left=1cm,right=1cm,bottom=1cm} }
		{\geometry{top=1cm,left=1cm,right=1cm,bottom=1cm} }
	}

% Turn off header and footer
\pagestyle{empty}

% for tight centres (less spacing)
\newenvironment{tightcenter}{%
  \setlength\topsep{0.5pt}
  \setlength\parskip{0.5pt}
  \begin{center}
}{%
  \end{center}
}

% Redefine section commands to use less space
\makeatletter
\renewcommand{\section}{\@startsection{section}{1}{0mm}%
                                {-1ex plus -.5ex minus -.2ex}%
                                {0.5ex plus .2ex}%x
                                {\normalfont\large\bfseries}}
\renewcommand{\subsection}{\@startsection{subsection}{2}{0mm}%
                                {-1explus -.5ex minus -.2ex}%
                                {0.5ex plus .2ex}%
                                {\normalfont\normalsize\bfseries}}
\renewcommand{\subsubsection}{\@startsection{subsubsection}{3}{0mm}%
                                {-1ex plus -.5ex minus -.2ex}%
                                {1ex plus .2ex}%
                                {\normalfont\small\bfseries}}
% change font
%\renewcommand{\familydefault}{\sfdefault}
%\renewcommand\rmdefault{\sfdefault}
\linespread{1.05}

\makeatother

% Define BibTeX command
\def\BibTeX{{\rm B\kern-.05em{\sc i\kern-.025em b}\kern-.08em
    T\kern-.1667em\lower.7ex\hbox{E}\kern-.125emX}}

% Don't print section numbers
\setcounter{secnumdepth}{0}

\setlength{\parindent}{0pt}
\setlength{\parskip}{0pt plus 0.5ex}

%% this changes all items (enumerate and itemize, reduce margins) ITEMIZE SEPARATION HERE
\setlength{\leftmargini}{0.5cm}
\setlength{\leftmarginii}{0.5cm}
\setlist[itemize,1]{leftmargin=2mm,labelindent=1mm,labelsep=1mm, itemsep = 0mm}
\setlist[itemize,2]{leftmargin=4mm,labelindent=1mm,labelsep=1mm, itemsep = 0mm}
%itemsep = 0mm
%\setlist{nosep}

% Need Logo Picture
%Watermark Top Right
%\usepackage{atbegshi,picture}
%\AtBeginShipout{\AtBeginShipoutUpperLeft{%
 % \put(\dimexpr\paperwidth-1.2cm\relax, -1.2cm){\makebox[0pt][r]{\framebox{\includegraphics[width = 0.3cm]{mountainbooks} Ger Teck}}}%
%}}

% Justify Package
\usepackage[document]{ragged2e}

% -------------------------------------------------------------------------------

% START OF DOCUMENT HERE

\begin{document}
\raggedright
\footnotesize
\begin{multicols*}{3}



% multicol parameters
% These lengths are set only within the two main columns
%\setlength{\columnseprule}{0.25pt}
\setlength{\premulticols}{1pt}
\setlength{\postmulticols}{1pt}
\setlength{\multicolsep}{1pt}
\setlength{\columnsep}{2pt}

%% DOCUMENT NAME HERE
\begin{center}
     \Large{\textbf{CS2100 Comp Org Notes}} \\
\end{center}
AY23/24 Sem 1, github.com/gerteck

\section{12. Boolean Algebra}
\subsubsection{Digital Circuits}
\begin{itemize}
\item Two voltage levels, 1 for high, 0 for low.
\item Digital circuits over analog circuits are more reliable, specified accurarcy (determinable).
\item Digital circuits abstracted using simple mathematical model: \textbf{(Boolean Algebra)}
\item Design, Analysis and simplification of digital circuit: \textbf{Digital Logic Design.}
\item \textbf{Combinational}: no memory, output depends solely on the input. (gates, adders, multiplexers)
\item \textbf{Sequential:} with memory, output depends on both input and current state. (counters, registers, memories)
\end{itemize}

\subsubsection{Boolean Algebra}
connectives in order of precedence:
\begin{itemize}
\item \textbf{negation} $A'$ equivalent to \textbf{NOT}
\item \textbf{conjunction} $A \cdot B $ equivalent to \textbf{AND}
\item \textbf{disjunction} $ A + B $ equivalent to \textbf{OR}
\item Note: always write the AND operator $ \cdot $, do not omit, as it may be confused with a 2 bit value, $AB$.
\item \textbf{Truth Table}: Provides listing of every possible combination of inputs and corresponding outputs. We may prove using truth table by comparing columns.
\end{itemize}

\subsubsection{Duality}
\begin{itemize}
\item \textbf{Duality}: if the AND/OR operators and identity elements 0/1 interchanged in a boolean equation, it remains valid.
\item e.g. the dual equation of $a+(b\cdot c)=(a+b)\cdot(a+c)$ is  $a\cdot(b+c)=(a\cdot b)+(a\cdot c).$, where if one is valid, then its dual is also valid.
\end{itemize}



\subsubsection{Laws \& Theorems of Boolean Algebra}
\centerline{\includegraphics[width=1\linewidth]{BAlaw1}}
\bigskip
\centerline{\includegraphics[width=1\linewidth]{BAlaw2}}
left/right equations are duals of each other

\subsubsection{Proving Theorems}
\begin{itemize}
\item Theorems can be proved using truth table, or by 
algebraic manipulation using other theorems/laws.
\end{itemize}
\centerline{\includegraphics[width=0.5\linewidth]{prove}}

\subsubsection{Boolean Functions, Complements}
\begin{itemize}
\item Represented by $F$, e.g. $F1(x,y,z) = x\cdot y\cdot z'$.
\item To prove $F1 = F2$, we may use boolean algebra, or use truth tables.
\item Complement Function is denoted as $F'$, obtained by interchanging 1 with 0 in function's output values.
\end{itemize}

% \vfill \null
\columnbreak

\subsubsection{Standard Forms}
\begin{itemize}
\item \textbf{Literals}: A Boolean variable on its own or in its complemented form. (e.g. $x$, $x'$)
\item \textbf{Product Term}: A single literal or a logical product (AND, $\cdot$) of several literals. (e.g. $x$, $x \cdot y \cdot z'$ )
\item \textbf{Sum Term}: A single literal or a logical sum (OR $+$) of several literals. (e.g. $A + B'$)
\item \textbf{sum-of-products (SOP) expression}: A product term or a logical sum (OR $+$) of several product terms.
\item \textbf{product-of-sums (POS) expression}: A sum term or a logical product (AND) of several sum terms.
\item Every boolean expr can be expressed in SOP/POS form.
\end{itemize}

\subsubsection{Minterms and Maxterms}
\begin{itemize}
\item \textbf{minterm} (of n variables): a product term that contains n literals from all the variables; denoted $m0$ to $m[2^n - 1]$
\item \textbf{maxterm} (of n variables): a sum term that contains n literals from all the variables; denoted $M0$ to $M[2^n - 1]$ 
\item Each minterm is the complement ($m2' = M2$) of its corresponding maxterm, vice versa.
\end{itemize}
\centerline{\includegraphics[width=0.5\linewidth]{minmax}}

\subsubsection{Canonical Forms}
\begin{itemize}
\item Canonical/normal form: a unique form of representation.
\item \textbf{Sum-of-minterms} = Canonical sum-of-products
\item \textbf{Product-of-maxterms} = Canonical product-of-sums
\end{itemize}
\centerline{\includegraphics[width=1\linewidth]{canon}}
• We can convert between sum-of-minterms and product-of-maxterms easily, by DeMorgan's.

\section{13. Logic Gates \& Simplification}
\subsubsection{Logic Gates}
\begin{itemize}
\item Fan-in: The number of inputs of a gate $\geq$ 1, 2. 
\item Implement bool exp / function as logic circuit.
\end{itemize}

\subsubsection{Universal Gates}
\begin{itemize}
\item \textbf{universal gate}: can implement a complete set of logic.
\item $\{AND, OR, NOT\}$ are a complete set of logic, sufficient for building any boolean function.
\item $\{NAND\}$ and $\{NOR\}$ themselves a complete set of logic. Implement NOT/AND/OR using only NAND or NOR gates.
\end{itemize}

\subsubsection{SOP and POS}
\begin{itemize}
\item an SOP expression can be easily implemented using \\
◦ 2-level AND-OR circuit or 2-level NAND circuit
\item a POS expression can be easily implemented using \\
◦ 2-level OR-AND circuit or 2-level NOR circuit
\end{itemize}

\subsubsection{Algebraic Simplification}
\begin{itemize}
\item \textbf{Function Simplification}: Make use of alegbraic (using theorems) or Karnaugh Maps (easier to use, limited to no more than 6 variables) or Quine-McCluskey.
\item \textbf{Algebraic Simplification}: aims to minimise \\
1. number of literals (prioritised over number of terms) \\
2. number of terms.
\end{itemize}

\subsubsection{Half Adder}
\begin{itemize}
\item Half adder is a circuit that adds 2 single bits (X, Y) to produce a result of 2 bits (C, S).
\centerline{\includegraphics[width=1\linewidth]{halfadder}}
\end{itemize}

\section{Universal Gates}
\centerline{\includegraphics[width=1\linewidth]{logicgates1}}
\bigskip
\centerline{\includegraphics[width=0.9\linewidth]{logicgates2}}

\columnbreak

\subsubsection{NAND as Universal Gate (Complete Logic Set)}
\centerline{\includegraphics[width=0.75\linewidth]{NAND}}
\smallskip
\subsubsection{NOR as Universal Gate  (Complete Logic Set)}
\centerline{\includegraphics[width=0.75\linewidth]{NOR}}


\subsubsection{Gray Code}
\begin{itemize}
\item Only a \textbf{single bit change} from one code value to the next. 4 bit standard gray code:
\centerline{\includegraphics[width=0.9\linewidth]{graycode}}
\item not restricted to decimal digits: $n$ bits can have up to $2^n$ values.
\item aka reflected binary code. To generate gray code, reflect.
\item not unique - multiple possible Gray code sequences
\end{itemize}

\subsubsection{K Maps}
\begin{itemize}
\item Simplify (SOP) expressions, with fewest possible product terms and literals.
\item Based on \textbf{Unifying Theorem ($A + A' = 1$), complement law.}
\item Abstract form of Venn diagram, matrix of squares, each square represents a \textbf{minterm}.
\item Two adjacent squares represent minterms that differ by exactly one literal.
\centerline{\includegraphics[width=1\linewidth]{kmap}}
\end{itemize}
\subsubsection{K Map for a function:}
\begin{itemize}
\item The K-map for a function is filled by putting: \\
◦ A \textbf{‘1’} in the square the corresponds to a \textbf{minterm} \\
◦ A ‘0’ otherwise
\item Each \textbf{valid grouping} of adjacent cells containing '1' corresponds to a
simpler product term. 
\item Group must have width/length (size) in \textbf{powers of 2}.
\item \textbf{larger group} = fewer literals in result product term
\item \textbf{fewer groups} = fewer product terms in final SOP exp.
\item Group maximum cells, and select fewest groups.
\end{itemize}


\vfill \null
\columnbreak

\section{K-Maps}
\subsubsection{3-Variable}
\centerline{\includegraphics[width=0.7\linewidth]{3variable}}

\subsubsection{4-Variable}
\centerline{\includegraphics[width=0.8\linewidth]{4variable}}

\subsubsection{5-Variable}
\centerline{\includegraphics[width=1\linewidth]{5variable}}

\subsubsection{Valid Groupings}
\centerline{\includegraphics[width=1\linewidth]{validgroup}}

\subsubsection{6-Variable}
\centerline{\includegraphics[width=0.9\linewidth]{6variable}}


\subsubsection{Using a K-map}
\begin{itemize}
\item K-map of function easily filled in when function in sum-of-minterms form. 
\item If not in sum-of-minterms, convert into sum-of-products (SOP) form, expand SOP expr into sum-of-minterms, or fill directly based on SOP.
\end{itemize}

\subsubsection{(E)PIs}
\begin{itemize}
\item \textbf{implicant}: product term that could be used to cover minterms of the function.
\item \textbf{prime implicant}: a product term obtained by combining the maximum possible number of minterms from adjacent squares in the map.
\item \textbf{essential prime implicant}: a prime implicant that includes at least one minterm that is not covered by any other prime implicant
\end{itemize}

\subsubsection{K-maps to find POS}
\begin{itemize}
\item shortcut: group maxterms (0s) of given function
\item long way: 1. convert K-map of F to K-map of F' (by flipping 0/1s), 2. get SOP of F' POS=(SOP)'.
\end{itemize}

\subsubsection{Don't-Care Conditions}
• denoted $d$, e.g.: $F(A, B, C) = \sum m(3, 5, 6) + \sum d(0, 7)$



\section{14. Combinational Circuits, MSI Components}
\subsubsection{Combinational Circuits}
\begin{itemize}
\item Two classes of logic circuits, combinational and sequential.
\centerline{\includegraphics[width=0.8\linewidth]{logiccircuits}}
\item \textbf{Function analysis of combinational circuit (CC)}: Label inputs and outputs, obtain functions of intermediate points and draw the truth table. Deduce functionality.
\item \textbf{CC design methods}: gate-level (with logic gates) and block-level (with functional blocks, e.g. IC chip).
\item Goals: reduce cost, increase speed, design simplicity.
\end{itemize}

\subsubsection{Gate-Level (SSI) Design}
\begin{itemize}
\item Design procedure: \\
1. State problem, label input output of circuit.\\
2. Draw truth table, obtain simplified boolean function. \\
3. Draw logic diagram.
\end{itemize}
\centerline{\includegraphics[width=1\linewidth]{buildha}}
\bigskip
\centerline{\includegraphics[width=1\linewidth]{buildfa}}

\subsubsection{Circuit Delays}
\begin{itemize}
\item Given a \textbf{logic gate} with delay $t$. If inputs stable at times $t1, t2, …, tn,$ then earliest time in which 
output will be stable is: $max( t1, t2, …, tn ) + t$ 
\end{itemize}
\centerline{\includegraphics[width=0.8\linewidth]{circuitdelay}}

\begin{itemize}
\item delay of a combinational circuit: repeat for all gates
\item E.g. n-bit parallel adder will have delay of:
	\begin{itemize}
	\item $S_n = ( (n – 1) * 2 + 2 ) t $
	\item $C_{n+1} = ( (n – 1) * 2 + 3 ) t $
	\item max delay = $( (n – 1)* 2 + 3 ) t$
	\end{itemize}
\end{itemize}



\vfill \null
\columnbreak

\section{15. MSI Components}

































  
\end{multicols*}
\end{document}